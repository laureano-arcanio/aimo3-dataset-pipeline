\documentclass[11pt]{article}

\usepackage[T1]{fontenc}
\usepackage[utf8]{inputenc}
\usepackage{amsmath,amssymb}
\usepackage[top=2.3cm]{geometry}

\begin{document}
\sloppy

\title{\Huge\bfseries Alice and Charlie's Garden Path Problem}
\author{\Large Laureano Arcanio \quad \large (February 2026)}
\date{}
\maketitle
\section*{Problem}


Alice and Charlie are observing a square garden divided into $1024\times 1024$ unit plots arranged in rows and columns.
Each plot contains either a \emph{rose} or a \emph{tulip}, arranged in a checkerboard pattern so that any two plots sharing a side contain different flowers.

\section*{Solution}

\subsection*{Step 1: Balance is automatic}

Along any allowed trail, each step moves to a neighboring plot.
Because the garden is colored in a checkerboard pattern, each step switches the type of flower encountered.
Thus the flowers along a trail strictly alternate.

If a trail contains an even number of plots, it visits the same number of roses and tulips.
If it contains an odd number of plots, the difference is exactly one.
Hence every allowed trail is automatically balanced, and the balance condition imposes no additional restriction.

Therefore, the problem reduces to determining the minimum number of allowed trails needed to cover all plots.

\subsection*{Step 2: An explicit construction}

Alice assigns one trail to each column of the field.
Each trail starts at the top plot of its column and proceeds straight downward through all $1024$ plots in that column.

These $1024$ trails:
\begin{itemize}
\item obey the movement rules,
\item are balanced by Step~1, and
\item together cover every plot exactly once.
\end{itemize}

Thus $M\le 1024$.

\subsection*{Step 3: Why fewer trails are impossible}

To show that Bob cannot do better, Charlie argues as follows.

Each plot is divided along its diagonal from the upper-left corner to the lower-right corner, producing two triangular half-plots.
Think of these diagonals as mirrors.

Light beams are sent into the field through the midpoints of all boundary edges, perpendicular to the boundary.
There are exactly $4\cdot 1024$ such entry points.
Each beam reflects off the diagonal mirrors and eventually exits the boundary, thereby pairing the boundary edges into $2\cdot 1024$ disjoint pairs.

Consider how these beams interact with Bob’s trails.
Whenever a beam passes through a region corresponding to a segment of a trail, it exits that region on the same type of boundary edge (horizontal or vertical) on which it entered.
A beam can change from a horizontal boundary to a vertical boundary, or vice versa, only by passing through a triangular half-plot that is not paired with another half-plot by the trail structure.

Each trail has exactly two endpoints.
Each endpoint creates exactly one such unpaired half-plot.
Therefore the total number of unpaired half-plots is exactly $2M$.

On the other hand, among the $2\cdot 1024$ boundary pairs, at least $1024$ pairs must connect a horizontal boundary edge with a vertical boundary edge.
Each such mixed pair forces a beam to pass through at least one unpaired half-plot.

Hence
\[
1024 \le 2M.
\]

Repeating the same argument after shifting all beam entry points by half a unit along the boundary produces an independent pairing and yields the same inequality again.
Adding the two inequalities gives
\[
2\cdot 1024 \le 2M,
\]
so
\[
M\ge 1024.
\]

\subsection*{Conclusion}

Bob has exhibited a covering using $1024$ trails, and Marley has shown that no covering can use fewer.
Therefore the minimum possible number of trails is
\[
M=1024.
\]

\section*{Answer}

\[
\boxed{1024}
\]

\end{document}
