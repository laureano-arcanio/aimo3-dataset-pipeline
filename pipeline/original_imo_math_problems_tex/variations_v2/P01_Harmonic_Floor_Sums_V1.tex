\documentclass[11pt]{article}

\usepackage[T1]{fontenc}
\usepackage[utf8]{inputenc}
\usepackage{amsmath,amssymb}
\usepackage[top=2.3cm]{geometry}

\begin{document}
\sloppy

\title{\Huge\bfseries Harmonic Floor Sums and Divisibility}
\author{\Large Laureano Arcanio \quad \large (February 2026)}
\date{}
\maketitle
\section*{Problem}


For a real number \(\alpha\) and each positive integer \(n\), define
\[
A_n(\alpha)=\sum_{k=1}^n \lfloor k\alpha\rfloor.
\]
Call \(\alpha\) \textbf{balanced} if \(A_n(\alpha)\) is divisible by \(n\) for every positive integer \(n\).

Let \(B=246{,}000\). Determine the number of balanced real numbers \(\alpha\) in the interval \([0,B]\).

\section*{Solution 1 (Direct / Intended)}

We first determine all balanced real numbers \(\alpha\).

Write \(\alpha=m+\varepsilon\) where \(m\in\mathbb{Z}\) and \(0\le \varepsilon<1\).
For each positive integer \(k\),
\[
\lfloor k\alpha\rfloor=\lfloor k(m+\varepsilon)\rfloor=km+\lfloor k\varepsilon\rfloor,
\]
hence
\[
A_n(\alpha)=\sum_{k=1}^n (km+\lfloor k\varepsilon\rfloor)
= m\sum_{k=1}^n k + \sum_{k=1}^n \lfloor k\varepsilon\rfloor
= m\cdot \frac{n(n+1)}2 + A_n(\varepsilon),
\]
where \(A_n(\varepsilon)=\sum_{k=1}^n \lfloor k\varepsilon\rfloor\).

\subsection*{Step 1: \(m\) is even.}
Using \(n=2\), the balanced condition gives \(2\mid A_2(\alpha)\), i.e.
\[
2 \mid \lfloor \alpha\rfloor+\lfloor 2\alpha\rfloor.
\]
Since \(\lfloor \alpha\rfloor=m\) and \(\lfloor 2\alpha\rfloor=\lfloor 2m+2\varepsilon\rfloor=2m+\lfloor 2\varepsilon\rfloor\),
we get
\[
A_2(\alpha)=3m+\lfloor 2\varepsilon\rfloor \equiv m+\lfloor 2\varepsilon\rfloor \pmod 2.
\]
Thus
\[
m+\lfloor 2\varepsilon\rfloor \equiv 0 \pmod 2. \tag{1}
\]
If \(m\) is odd, then \(\lfloor 2\varepsilon\rfloor=1\), so \(\varepsilon\ge \tfrac12\).

Now use \(n=4\). Balanced means \(4\mid A_4(\alpha)\).
But
\[
A_4(\alpha)= m\cdot \frac{4\cdot 5}{2} + A_4(\varepsilon) = 10m + A_4(\varepsilon).
\]
If \(m\) is odd, then \(10m\equiv 2 \pmod 4\), hence we must have
\[
A_4(\varepsilon)\equiv 2 \pmod 4. \tag{2}
\]
On the other hand, \(\varepsilon\ge \tfrac12\) implies
\[
\lfloor \varepsilon\rfloor=0,\quad \lfloor 2\varepsilon\rfloor\ge 1,\quad \lfloor 3\varepsilon\rfloor\ge 1,\quad \lfloor 4\varepsilon\rfloor\ge 2,
\]
so
\[
A_4(\varepsilon)=\lfloor \varepsilon\rfloor+\lfloor 2\varepsilon\rfloor+\lfloor 3\varepsilon\rfloor+\lfloor 4\varepsilon\rfloor
\ge 0+1+1+2=4,
\]
and in fact for all \(\varepsilon\in[\tfrac12,1)\) one checks \(A_4(\varepsilon)\in\{4,5,6\}\), hence
\(A_4(\varepsilon)\equiv 0,1,2\pmod 4\) but \emph{never} equals \(2\pmod 4\) together with the minimal constraints forced by (1).
A direct computation avoids any ambiguity: for \(\varepsilon\in[\tfrac12,\tfrac23)\),
\((\lfloor \varepsilon\rfloor,\lfloor2\varepsilon\rfloor,\lfloor3\varepsilon\rfloor,\lfloor4\varepsilon\rfloor)=(0,1,1,2)\) so \(A_4(\varepsilon)=4\equiv 0\);
for \(\varepsilon\in[\tfrac23,\tfrac34)\), this tuple is \((0,1,2,2)\) so \(A_4(\varepsilon)=5\equiv 1\);
for \(\varepsilon\in[\tfrac34,1)\), it is \((0,1,2,3)\) so \(A_4(\varepsilon)=6\equiv 2\).
But the last range \([\tfrac34,1)\) also gives \(\lfloor 3\varepsilon\rfloor=2\), and then taking \(n=3\) (balanced means \(3\mid A_3(\alpha)\))
forces a contradiction as follows:
\[
A_3(\alpha)=m\cdot \frac{3\cdot4}{2}+A_3(\varepsilon)=6m + (\lfloor \varepsilon\rfloor+\lfloor2\varepsilon\rfloor+\lfloor3\varepsilon\rfloor).
\]
With \(\varepsilon\in[\tfrac34,1)\), \(A_3(\varepsilon)=0+1+2=3\), so \(A_3(\alpha)\equiv 0 \pmod 3\) holds automatically,
but then (2) requires \(A_4(\varepsilon)\equiv 2\) which only occurs in this range; combining with (1) we already have \(m\) odd.
Now take \(n=5\). Then
\[
A_5(\alpha)=m\cdot \frac{5\cdot6}{2}+A_5(\varepsilon)=15m + A_5(\varepsilon)\equiv 0\pmod 5.
\]
If \(m\) is odd then \(15m\equiv 0\pmod 5\), so we must have \(A_5(\varepsilon)\equiv 0\pmod 5\).
But for \(\varepsilon\in[\tfrac34,1)\),
\[
(\lfloor \varepsilon\rfloor,\lfloor2\varepsilon\rfloor,\lfloor3\varepsilon\rfloor,\lfloor4\varepsilon\rfloor,\lfloor5\varepsilon\rfloor)=(0,1,2,3,3 \text{ or }4),
\]
so \(A_5(\varepsilon)\in\{9,10\}\), hence \(A_5(\varepsilon)\equiv 4,0\pmod 5\). The case \(A_5(\varepsilon)\equiv 0\) would force \(A_5(\varepsilon)=10\),
which requires \(\lfloor 5\varepsilon\rfloor=4\), i.e. \(\varepsilon\in[\tfrac45,1)\).
Then \(A_4(\varepsilon)=6\equiv2\) still, but now check \(n=6\):
\[
A_6(\varepsilon)=0+\lfloor2\varepsilon\rfloor+\cdots+\lfloor6\varepsilon\rfloor \ge 1+2+3+4+4 =14,
\]
and in fact in \([\tfrac45,1)\) one gets \(A_6(\varepsilon)\equiv 2,3,4 \pmod 6\), never \(0\), contradicting balance for \(n=6\).
Therefore \(m\) cannot be odd, so \(m\) is even.

(Any short parity-based route is acceptable; the essential conclusion is \(m\) even.)

\subsection*{Step 2: \(\varepsilon=0\).}
Now assume \(m\) is even. Then for every \(n\),
\[
m\cdot \frac{n(n+1)}2 \equiv 0 \pmod n,
\]
so the balance condition implies
\[
A_n(\varepsilon)=\sum_{k=1}^n \lfloor k\varepsilon\rfloor \equiv 0 \pmod n
\quad\text{for all } n. \tag{3}
\]
If \(\varepsilon>0\), let \(t\) be the smallest positive integer such that \(\lfloor t\varepsilon\rfloor\ge 1\).
Then for \(1\le k\le t-1\), \(\lfloor k\varepsilon\rfloor=0\), and by minimality \(\lfloor t\varepsilon\rfloor=1\).
Hence
\[
A_t(\varepsilon)=\sum_{k=1}^t \lfloor k\varepsilon\rfloor = 1,
\]
contradicting (3) since \(t\ge 1\) would require \(t\mid 1\), impossible for \(t>1\).
Thus \(\varepsilon=0\).

Therefore every balanced \(\alpha\) is an even integer, and every even integer is balanced.

\subsection*{Step 3: Count balanced numbers in \([0,B]\).}
With \(B=246{,}000\), the balanced numbers in \([0,B]\) are
\[
0,2,4,\dots,199{,}996 = 2\cdot 123{,}000.
\]
There are \(123{,}000-0+1=123{,}001\) such numbers.

\section*{Answer}
\[
\boxed{123{,}001}
\]

\end{document}
