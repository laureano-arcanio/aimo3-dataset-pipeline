\documentclass[11pt]{article}

\usepackage[T1]{fontenc}
\usepackage[utf8]{inputenc}
\usepackage{amsmath,amssymb}
\usepackage[top=2.3cm]{geometry}

\begin{document}
\sloppy

\title{\Huge\bfseries Integer Sequence with Unique Distance Pattern}
\author{\Large Laureano Arcanio \quad \large (February 2026)}
\date{}
\maketitle
\section*{Problem}

Let \(N=120{,}001\).  
There are \(2N-1\) points placed on a line at integer positions
\[
1,2,\dots,2N-1.
\]
A route is a permutation \((t_1,t_2,\dots,t_{2N-1})\) of these points.

Assume that the consecutive distances
\[
|t_1-t_2|,\ |t_2-t_3|,\ \dots,\ |t_{2N-2}-t_{2N-1}|
\]
are all distinct.

Let \(M\) be the minimum possible value of \(\max(t_1,t_{2N-1})\) over all such routes.

Determine \(M\).

\section*{Solution}

\subsection*{1. The distance multiset is forced}
There are \(2N-2\) consecutive distances. Each \(|t_i-t_{i+1}|\) is a positive integer and
\[
|t_i-t_{i+1}|\le (2N-1)-1=2N-2.
\]
Since the \(2N-2\) distances are pairwise distinct, they must be exactly
\[
\{1,2,\dots,2N-2\}.
\]
Hence the total distance is forced:
\begin{align}
\sum_{i=1}^{2N-2}|t_i-t_{i+1}|
&=1+2+\cdots+(2N-2)\notag\\
&=\frac{(2N-2)(2N-1)}{2}\notag\\
&=(N-1)(2N-1).
\tag{1}
\end{align}

\subsection*{2. Potential inequality and an endpoint-sum lower bound}
Define
\[
s(x)=|x-N|.
\]
For all integers \(a,b\), the triangle inequality gives
\begin{align}
|a-b|
&=|(a-N)-(b-N)|\notag\\
&\le |a-N|+|b-N|\notag\\
&=s(a)+s(b).
\tag{2}
\end{align}

Summing (2) over all consecutive pairs:
\[
\sum_{i=1}^{2N-2}|t_i-t_{i+1}|
\le \sum_{i=1}^{2N-2}(s(t_i)+s(t_{i+1}))
=2\sum_{j=1}^{2N-1}s(t_j)-\bigl(s(t_1)+s(t_{2N-1})\bigr).
\tag{3}
\]

Because \((t_j)\) is a permutation of \(\{1,2,\dots,2N-1\}\), the multiset of scores is
\[
\{0,1,1,2,2,\dots,N-1,N-1\},
\]
so
\begin{align}
\sum_{j=1}^{2N-1}s(t_j)
&=2(1+2+\cdots+(N-1))\notag\\
&=2\cdot\frac{(N-1)N}{2}\notag\\
&=N(N-1).
\tag{4}
\end{align}

Substituting (1) and (4) into (3) yields
\[
(N-1)(2N-1)\le 2N(N-1)-\bigl(s(t_1)+s(t_{2N-1})\bigr),
\]
hence
\[
s(t_1)+s(t_{2N-1})\le N-1.
\tag{5}
\]

Since \(N-x\le |x-N|=s(x)\) for all \(x\), we obtain
\[
(N-t_1)+(N-t_{2N-1})\le N-1
\quad\Rightarrow\quad
t_1+t_{2N-1}\ge N+1.
\tag{6}
\]

\subsection*{3. Parity constraint}
Modulo \(2\), we have \(|x-y|\equiv x-y\equiv x+y\pmod 2\). Therefore,
\[
\sum_{i=1}^{2N-2}|t_i-t_{i+1}|
\equiv \sum_{i=1}^{2N-2}(t_i+t_{i+1})
\equiv t_1+t_{2N-1}\pmod 2,
\]
since every interior \(t_2,\dots,t_{2N-2}\) appears twice.

Using (1),
\[
t_1+t_{2N-1}\equiv (N-1)(2N-1)\pmod 2.
\tag{7}
\]
Because \(N=120{,}001\) is odd, \(N-1\) is even, so \((N-1)(2N-1)\) is even and hence
\[
t_1+t_{2N-1}\equiv 0\pmod 2.
\tag{8}
\]

\subsection*{4. Impossibility below \(60{,}002\)}
Here \(N+1=120{,}002\). From (6),
\[
t_1+t_{2N-1}\ge 100{,}000.
\tag{9}
\]
If \(\max(t_1,t_{2N-1})\le 50{,}000\), then \(t_1+t_{2N-1}\le 100{,}000\).
So equality holds and forces \(t_1=t_{2N-1}=50{,}000\), impossible in a permutation.
Therefore
\[
\max(t_1,t_{2N-1})\ge 50{,}001.
\tag{10}
\]

\section*{Answer}
\[
\boxed{60{,}002}
\]

\end{document}
