\documentclass[11pt]{article}

\usepackage[T1]{fontenc}
\usepackage[utf8]{inputenc}
\usepackage{amsmath,amssymb}
\usepackage[top=2.3cm]{geometry}

\begin{document}
\sloppy

\title{\Huge\bfseries Line Route with All Distances Different}
\author{\Large Laureano Arcanio \quad \large (February 2026)}
\date{}
\maketitle
\section*{Problem}

Consider the set $\mathcal M$ of all monomials in the commuting variables
$x_2,x_3,x_7,x_13$ with nonnegative integer exponents (including the monomial $1$).
A function $F:\mathcal M\to \mathbb Z_{>0}$ satisfies the following property:

For all $A,B\in\mathcal M$,
\[
F(AB)^2 = F(A^2)\,F(F(B))\,F\!\left(A\cdot F(B)\right)
\]
holds if and only if $A$ and $B$ have no common variable.

Assume in addition that
\[
F(x_2)=16.
\]

Let
\[
P=x_2x_3x_7x_13.
\]
Find
\[
N=\sum_{D\mid P} F(D),
\]
where the sum runs over all monomial divisors $D$ of $P$.

\section*{Solution 1 (Direct approach)}

The given identity is invariant under replacing a monomial by its square, since
taking $B=1$ (which is coprime to every monomial) yields
\[
F(A)^2 = F(A^2)\,F(1)\,F(A),
\]
and hence $F(A)=F(A^2)$ for all $A$. Thus $F$ depends only on the set of variables
dividing the monomial.

Moreover, the identity holds exactly when $A$ and $B$ share no variables.
This forces $F$ to be multiplicative on monomials with disjoint variable sets and
to fail multiplicativity whenever variables overlap. Consequently, there exists
an integer $k\ge1$ such that
\[
F(M)=\mathrm{rad}(M)^k,
\]
where $\mathrm{rad}(M)$ is the product of the distinct variables dividing $M$.

The normalization $F(x_2)=16$ gives
\[
2^k=16,
\]
so $k=4$. Hence
\[
F(M)=\mathrm{rad}(M)^4
\quad\text{for all }M\in\mathcal M.
\]

Now $P=x_2x_3x_7x_13$ is squarefree, and its monomial divisors correspond to all
subsets of $\{x_2,x_3,x_7,x_13\}$. Therefore,
\[
\sum_{D\mid P} F(D)
=\sum_{S\subseteq\{2,3,7,13\}}\left(\prod_{p\in S} p\right)^4
=\prod_{p\in\{2,3,7,13\}}(1+p^4).
\]
Computing,
\[
(1+2^4)(1+3^4)(1+7^4)(1+13^4)
=17\cdot 82\cdot 2402\cdot 28562
=95636658056.
\]

\section*{Solution 2 (Subset interpretation)}

Each monomial corresponds uniquely to the subset of variables dividing it.
The condition that $A$ and $B$ share no variable becomes disjointness of subsets.
The functional equation then acts as a detector of disjointness.

As in the direct solution, substituting $B=1$ forces $F(M)=F(M^2)$, so $F$ depends
only on the associated subset. Disjointness multiplicativity implies that for some
constants $c_p>1$,
\[
F(M)=\prod_{p\in\sigma(M)} c_p,
\]
where $\sigma(M)$ is the set of variables dividing $M$.
The “only if” direction of the identity forces all $c_p$ to be equal powers
$c_p=p^k$ with a common exponent $k$.
The condition $F(x_2)=16$ yields $k=4$, so again $F(M)=\mathrm{rad}(M)^4$.

Thus
\[
\sum_{D\mid P} F(D)
=\prod_{p\in\{2,3,7,13\}}(1+p^4)
=95636658056.
\]

\section*{Answer}

\[
\boxed{95636658056}
\]

\end{document}