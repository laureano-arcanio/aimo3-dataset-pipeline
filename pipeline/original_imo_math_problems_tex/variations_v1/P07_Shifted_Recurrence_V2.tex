\documentclass[11pt]{article}

\usepackage[T1]{fontenc}
\usepackage[utf8]{inputenc}
\usepackage{amsmath,amssymb}
\usepackage[top=2.3cm]{geometry}

\begin{document}
\sloppy

\title{\Huge\bfseries Line Route with All Distances Different}
\author{\Large Laureano Arcanio \quad \large (February 2026)}
\date{}
\maketitle
\section*{Problem}

Let $(a_n)$ be a real sequence satisfying:

\begin{enumerate}
\item \emph{Eventually periodic:} There exist integers $N_0 \ge 1$ and $T \ge 1$ such that
\[
a_{n+T} = a_n \quad \text{for all } n \ge N_0.
\]
\item \emph{Recurrence:} For all integers $n \ge 7$,
\[
a_{n+2} + a_{2n} = a_{n+1} + a_{2n+1}.
\]
\item \emph{Square-summable increments:}
\[
\sum_{n=1}^{\infty} (a_{n+1} - a_n)^2 < \infty.
\]
\end{enumerate}

Determine the \emph{maximum possible} value of $N$, the number of distinct real values attained by the sequence $(a_n)$.

\section*{Solution 1 (Forcing the tail to be constant)}

Define the first differences
\[
d_n := a_{n+1}-a_n \qquad (n\ge 1).
\]
From the recurrence, for every $n\ge 7$,
\[
a_{n+2}-a_{n+1} = a_{2n+1}-a_{2n},
\]
hence
\[
d_{n+1} = d_{2n} \qquad (n\ge 7).
\]
Equivalently, for every integer $x\ge 8$ (letting $x=n+1$),
\begin{equation}
d_x = d_{2(x-1)}. \tag{1}
\end{equation}

Let $h(x)=2(x-1)$. Iterating \eqref{1} gives, for all $k\ge 0$ and $x\ge 2$,
\[
d_x = d_{h^k(x)}.
\]
We claim that for all $k\ge 0$ and $x\ge 2$,
\begin{equation}
h^k(x)=2^k(x-2)+2. \tag{2}
\end{equation}
This is proved by induction on $k$. For $k=0$, $h^0(x)=x=2^0(x-2)+2$. Assume true for $k$.
Then
\[
h^{k+1}(x)=h(h^k(x))=2(h^k(x)-1)
=2\bigl(2^k(x-2)+2-1\bigr)=2^{k+1}(x-2)+2,
\]
proving \eqref{2}.

Fix any $x\ge 8$. Then by \eqref{2}, the integers $h^k(x)=2^k(x-2)+2$ are strictly increasing in $k$,
hence pairwise distinct. Since $d_{h^k(x)}=d_x$ for all $k\ge 0$, if $d_x\neq 0$ then
\[
\sum_{n=1}^{\infty} d_n^2 \;\ge\; \sum_{k=0}^{\infty} d_{h^k(x)}^2
=\sum_{k=0}^{\infty} d_x^2 = \infty,
\]
contradicting $\sum_{n=1}^{\infty} d_n^2<\infty$. Therefore,
\[
d_n=0 \quad \text{for all } n\ge 8.
\]
Hence
\[
a_{8}=a_{9}=a_{10}=\cdots =: C
\]
for some real constant $C$.

Thus all values of the sequence lie in the set $\{a_1,a_2,\dots,a_7,C\}$, so the number of distinct
values satisfies
\[
N \le 8.
\]

\section*{Solution 2 (Independent construction and optimality)}

We independently determine all sequences satisfying the hypotheses and count the number
of distinct values they may attain.

Assume $(a_n)$ satisfies the three given conditions. From the recurrence,
\[
a_{n+2}-a_{n+1} = a_{2n+1}-a_{2n},
\]
and the square-summability of $(a_{n+1}-a_n)$, the same iteration argument as above
forces
\[
a_{n+1}=a_n \quad \text{for all } n\ge 8.
\]
Hence there exists a real constant $C$ such that $a_n=C$ for all $n\ge 8$.

Conversely, choose arbitrary real numbers $a_1,a_2,\dots,a_7,C$ and define
\[
a_n:=C \quad \text{for all } n\ge 8.
\]
Then $(a_n)$ is eventually constant, hence eventually periodic. The increment sequence
satisfies
\[
(a_{n+1}-a_n)^2 = 0 \quad \text{for all } n\ge 8,
\]
so the square-sum condition holds. The recurrence is easily checked directly:
for all $n\ge 7$, all involved terms equal $C$.

Therefore, every solution has all its values in $\{a_1,a_2,\dots,a_7,C\}$, and every such 8-tuple
produces a valid solution. Hence
\[
N\le 8.
\]

To show that $8$ is attainable, take $a_1,a_2,\dots,a_7,C$ pairwise distinct, for example
\[
a_1=14,\qquad a_2=12,\qquad a_3=10,\qquad a_4=8,\qquad a_5=6,\qquad a_6=4,\qquad a_7=2,\qquad C=0,
\]
and define $a_n=C$ for all $n\ge 8$. This sequence satisfies all conditions and attains
exactly 8 distinct values.

\section*{Answer}

The number of distinct values is always at most $8$, and this bound is sharp. Hence the
maximum possible value of $N$ is
\[
\boxed{8}.
\]

\end{document}
