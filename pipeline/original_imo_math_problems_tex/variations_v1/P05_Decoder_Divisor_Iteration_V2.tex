\documentclass[11pt]{article}

\usepackage[T1]{fontenc}
\usepackage[utf8]{inputenc}
\usepackage{amsmath,amssymb}
\usepackage[top=2.3cm]{geometry}

\begin{document}
\sloppy

\title{\Huge\bfseries Line Route with All Distances Different}
\author{\Large Laureano Arcanio \quad \large (February 2026)}
\date{}
\maketitle
\section*{Problem}

A ``decoder'' takes a positive integer $N$ (having at least three proper divisors) and outputs the sum of the three largest proper divisors of $N$.
Starting from $a_1$, define a sequence by
\[
a_{n+1} = \psi(a_n)\qquad (n \ge 1),
\]
where $\psi(N)$ denotes the sum of the three largest proper divisors of $N$.

Assume that the sequence is well-defined for all $n$ (that is, every $a_n$ has at least three proper divisors).

Let $M = 900{,}000$.
How many integers $a_1$ with $1 \le a_1 \le M$ satisfy this condition?

\section*{Solution 1 (classification first)}

From the analysis of the recurrence $a_{n+1}=\psi(a_n)$, one proves that the sequence is well-defined for all $n$ if and only if the initial value is of the form
\[
a_1 = 6\cdot 12^e \cdot \ell,
\]
where $e \ge 0$ and $\gcd(\ell,10)=1$.
For such numbers, the three largest proper divisors are eventually $\tfrac{x}{2},\tfrac{x}{3},\tfrac{x}{4}$, and the recurrence remains defined forever; for all other starting values, the process eventually fails.

Thus we must count all integers of the form $6\cdot 12^e \cdot \ell$ not exceeding $900{,}000$ with $\gcd(\ell,10)=1$.

The inequality $6\cdot 12^e \le 900{,}000$ gives $e\in\{0,1,2,3,4\}$, since
\[
6\cdot 12^4 = 124{,}416 \le 900{,}000 < 6\cdot 12^5 = 1{,}492{,}992.
\]
For each $e$, set
\[
L_e = \left\lfloor \frac{900{,}000}{6\cdot 12^e} \right\rfloor.
\]
We must count integers $\ell \le L_e$ that are not divisible by $2$ or $5$.
By inclusion--exclusion, this number equals
\[
L_e - \left\lfloor \frac{L_e}{2} \right\rfloor
      - \left\lfloor \frac{L_e}{5} \right\rfloor
      + \left\lfloor \frac{L_e}{10} \right\rfloor.
\]

Evaluating:
\[
\begin{array}{c|c|c}
e & L_e & \text{count} \\
\hline
0 & 150{,}000 & 60{,}000 \\
1 & 12{,}500 & 5{,}000 \\
2 & 1{,}041 & 417 \\
3 & 86 & 34 \\
4 & 7 & 3 \\
\end{array}
\]

Summing all cases gives
\[
60{,}000 + 5{,}000 + 417 + 34 + 3 = 65{,}454.
\]

\section*{Solution 2 (dynamics first)}

Consider the recurrence $a_{n+1}=\psi(a_n)$.
A divisor-ordering argument shows that if the iteration is to remain defined forever, every term must be divisible by $6$.
Moreover, the only sustainable configuration of the three largest proper divisors is
\[
\frac{x}{2},\ \frac{x}{3},\ \frac{x}{4},
\]
which yields
\[
\psi(x)=\frac{x}{2}+\frac{x}{3}+\frac{x}{4}=\frac{13}{12}x.
\]

This forces the initial value to have the rigid structure
\[
a_1 = 6\cdot 12^e \cdot \ell,
\]
and to prevent larger competing divisors, the factor $\ell$ must be coprime to $10$.
Conversely, any such $a_1$ keeps the iteration well-defined for all $n$ (and eventually stabilizes when $e=0$).

Therefore, the valid initial values $a_1 \le 900{,}000$ are exactly those counted in Solution~1, giving the same total.

\section*{Answer}

\[
\boxed{65{,}454}
\]

\end{document}
