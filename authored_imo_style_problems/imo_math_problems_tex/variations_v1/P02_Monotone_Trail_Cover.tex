\documentclass[11pt]{article}

\usepackage[T1]{fontenc}
\usepackage[utf8]{inputenc}
\usepackage{amsmath,amssymb}
\usepackage[top=2.3cm]{geometry}

\begin{document}
\sloppy

\title{\Huge\bfseries P02 Monotone Trail Cover}
\author{\Large Laureano Arcanio \quad \large (February 2026)}
\date{}
\maketitle
\section*{Problem}

A square grid of size $2049\times 2049$ is colored in a checkerboard pattern so that any two cells sharing a side have different colors.
Each cell is colored either \emph{black} or \emph{white}.

A \emph{trail} is a sequence of adjacent cells satisfying one of the following rules:
\begin{itemize}
\item the trail moves only to the right or downward, or
\item the trail moves only to the right or upward.
\end{itemize}

A trail is called \emph{balanced} if the number of black cells it visits differs from the number of white cells it visits by at most one.

We wish to cover the entire grid with trails so that every cell belongs to exactly one trail and every trail is balanced.
Determine the smallest possible number $M$ of trails needed.

\section*{Solution}

\subsection*{Step 1: Balance is automatic}

Along any allowed trail, each step moves to a neighboring cell.
Because the grid is colored in a checkerboard pattern, each step switches color.
Thus the colors along a trail strictly alternate.

If a trail contains an even number of cells, it visits the same number of black and white cells.
If it contains an odd number of cells, the difference is exactly one.
Hence every allowed trail is automatically balanced, and the balance condition imposes no additional restriction.

Therefore, the problem reduces to determining the minimum number of allowed trails needed to cover all cells.

\subsection*{Step 2: An explicit construction}

A construction assigns one trail to each column of the grid.
Each trail starts at the top cell of its column and proceeds straight downward through all $2049$ cells in that column.

These $2049$ trails:
\begin{itemize}
\item obey the movement rules,
\item are balanced by Step~1, and
\item together cover every cell exactly once.
\end{itemize}

Thus $M\le 2049$.

\subsection*{Step 3: Why fewer trails are impossible}

To show that no covering can use fewer trails, we argue as follows.

Each cell is divided along its diagonal from the upper-left corner to the lower-right corner, producing two triangular half-cells.
Think of these diagonals as mirrors.

Light beams are sent into the grid through the midpoints of all boundary edges, perpendicular to the boundary.
There are exactly $4\cdot 2049$ such entry points.
Each beam reflects off the diagonal mirrors and eventually exits the boundary, thereby pairing the boundary edges into $2\cdot 2049$ disjoint pairs.

Consider how these beams interact with the trails.
Whenever a beam passes through a region corresponding to a segment of a trail, it exits that region on the same type of boundary edge (horizontal or vertical) on which it entered.
A beam can change from a horizontal boundary to a vertical boundary, or vice versa, only by passing through a triangular half-cell that is not paired with another half-cell by the trail structure.

Each trail has exactly two endpoints.
Each endpoint creates exactly one such unpaired half-cell.
Therefore the total number of unpaired half-cells is exactly $2M$.

On the other hand, among the $2\cdot 2049$ boundary pairs, at least $2049$ pairs must connect a horizontal boundary edge with a vertical boundary edge.
Each such mixed pair forces a beam to pass through at least one unpaired half-cell.

Hence
\[
2049 \le 2M.
\]

Repeating the same argument after shifting all beam entry points by half a unit along the boundary produces an independent pairing and yields the same inequality again.
Adding the two inequalities gives
\[
2\cdot 2049 \le 2M,
\]
so
\[
M\ge 2049.
\]

\subsection*{Conclusion}

A covering using $2049$ trails exists, and the lower bound shows that no covering can use fewer.
Therefore the minimum possible number of trails is
\[
M=2049.
\]

\section*{Answer}

\[
\boxed{2049}
\]

\end{document}
