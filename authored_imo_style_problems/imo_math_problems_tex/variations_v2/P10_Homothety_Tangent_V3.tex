\documentclass[11pt]{article}

\usepackage[T1]{fontenc}
\usepackage[utf8]{inputenc}
\usepackage{amsmath,amssymb}
\usepackage[top=2.3cm]{geometry}

\begin{document}
\sloppy

\title{\Huge\bfseries P10 Homothety Tangent V3}
\author{\Large Laureano Arcanio \quad \large (February 2026)}
\date{}
\maketitle
\section*{Problem}


Let $ABC$ be a triangle with incenter $I$ and incircle $\omega$.  
Points $X,Y\in BC$ are defined as follows:
\begin{itemize}
\item the line through $X$ parallel to $AC$ is tangent to $\omega$;
\item the line through $Y$ parallel to $AB$ is tangent to $\omega$.
\end{itemize}
Let $AI$ meet the circumcircle of $\triangle ABC$ again at $P\neq A$.

For an integer $m\ge 2$, let $K_m\in AB$ and $L_m\in AC$ satisfy
\[
\frac{AK_m}{AB}=\frac{AL_m}{AC}=\frac{1}{m}.
\]
Define the statement $\mathcal{S}(m)$:

\emph{For every triangle $ABC$ with $AB<AC<BC$, the identity}
\[
\angle K_m I L_m + \angle YPX = 180^\circ
\]
\emph{holds.}

Let
\[
N=\#\{\,m\in\{2,3,\dots,200000\}\mid \mathcal{S}(m)\text{ is true}\,\}.
\]
Determine $N$.

\section*{Solution 1 (Direct / Homothety)}

Let $A_1$ be the reflection of $A$ across $I$.  
As in the original tangency construction, $A_1X$ and $A_1Y$ are tangents to $\omega$.  
Hence the quadrilaterals $B,P,A_1,X$ and $C,Y,A_1,P$ are cyclic.

Therefore,
\[
\angle APX=\angle A_1BC, \qquad
\angle YPA=\angle BCA_1,
\]
and
\[
\angle YPX=\angle A_1BC+\angle BCA_1.
\]

Thus the given identity becomes
\[
\angle K_m I L_m = \angle CA_1B.
\tag{1}
\]

Consider the homothety centered at $A$ with factor $m$.  
By construction, it sends $K_m$ to $B$ and $L_m$ to $C$.  
It sends $I$ to the point $A+m(I-A)$.

The equality \((1)\) holds for all triangles if and only if this image of $I$ coincides with $A_1$, which is the reflection of $A$ across $I$, i.e.
\[
A_1 = A+2(I-A).
\]
Hence $m=2$.

For $m=2$, the homothety sends $(K_2,L_2,I)$ to $(B,C,A_1)$, giving
\[
\angle K_2 I L_2 = \angle CA_1B,
\]
and the angle sum is $180^\circ$.  
For any $m\neq 2$, equality \((1)\) fails for a generic triangle.

Thus $\mathcal{S}(m)$ holds if and only if $m=2$.

\section*{Solution 2 (Invariance Argument)}

From the tangency and cyclicity alone (independent of $m$), we obtain
\[
\angle YPX = 180^\circ - \angle CA_1B,
\]
where $A_1$ is the reflection of $A$ across $I$.

Hence
\[
\angle K_m I L_m + \angle YPX = 180^\circ
\quad\Longleftrightarrow\quad
\angle K_m I L_m = \angle CA_1B.
\]

The right-hand side depends only on the fixed point $A_1$, while the left-hand side depends on the positions of $K_m,L_m$ on $AB,AC$.
The only value of $m$ for which the dilation centered at $A$ maps
\[
(K_m,L_m,I)\longmapsto(B,C,A_1)
\]
is $m=2$.

Therefore the equality can hold identically for all triangles only when $m=2$.

\section*{Answer}

\[
\boxed{N=1}
\]

\end{document}
