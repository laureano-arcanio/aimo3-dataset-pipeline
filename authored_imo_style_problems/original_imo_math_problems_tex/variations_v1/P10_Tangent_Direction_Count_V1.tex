\documentclass[11pt]{article}

\usepackage[T1]{fontenc}
\usepackage[utf8]{inputenc}
\usepackage{amsmath,amssymb}
\usepackage[top=2.3cm]{geometry}

\begin{document}
\sloppy

\title{\Huge\bfseries Line Route with All Distances Different}
\author{\Large Laureano Arcanio \quad \large (February 2026)}
\date{}
\maketitle
\section*{Problem}

Two circles $\Omega,\Gamma$ with centers $M,N$ intersect at $A,B$.  
The line $MN$ meets $\Omega$ again at $C$ and $\Gamma$ again at $D$, with $C,M,N,D$ in this order.

Let $P$ be the circumcenter of $\triangle ACD$.  
The line $AP$ meets $\Omega$ again at $E\neq A$ and $\Gamma$ again at $F\neq A$.  
Let $H$ be the orthocenter of $\triangle PMN$, and let $\mathcal{C}$ be the circumcircle of $\triangle BEF$.

For each integer $k$ in $0\le k\le 79{,}999$, let $\ell_k$ be the line through $H$ whose direction is obtained by rotating the \emph{direction} of $AP$ clockwise by an angle $\dfrac{k\pi}{80{,}000}$ (so $\ell_0$ is the line through $H$ parallel to $AP$).

Find the number of integers $k$ for which $\ell_k$ is tangent to $\mathcal{C}$.

\section*{Solution 1 (Direct)}

\subsection*{Step 1: The geometric crux (tangency of $\ell_0$)}
We record the following lemma, which is exactly the tangency conclusion of the underlying IMO 2025/2 configuration.

\medskip
\noindent\textbf{Lemma.} \emph{In the above configuration, the line through $H$ parallel to $AP$ is tangent to the circumcircle $\mathcal{C}$ of $\triangle BEF$.}
\medskip

\noindent\emph{Remark (what this lemma encapsulates).}
A full proof can be given by directed angles (or inversion) and proceeds via the standard rigidity chain:
(i) establish transport parallels $CE\parallel AD$ and $DF\parallel AC$,
(ii) define $A' = CE\cap DF$ and introduce the circumcenter $T$ of $\triangle A'EF$ so that $TE=TF$,
(iii) show $HT\parallel AP$ using the orthocenter relations in $\triangle PMN$ (equivalently $MH\parallel AD$ and $NH\parallel AC$),
(iv) use $TE=TF$ together with angle/power criteria to deduce tangency of the line parallel to $AP$ to $(BEF)$.
This is precisely the ``hard part'' of the original problem.

\subsection*{Step 2: Reduce to a counting statement}
By definition, $\ell_0$ is the line through $H$ parallel to $AP$, hence by the Lemma $\ell_0$ is tangent to $\mathcal{C}$.

Now consider any $k\neq 0$. Then $\ell_k$ is obtained by rotating the direction of $AP$ by a nonzero angle $\frac{k\pi}{80000}$, so $\ell_k$ is \emph{not} parallel to $AP$. Therefore $\ell_k\neq \ell_0$ as lines through the same point $H$.

But a fixed circle $\mathcal{C}$ has at most two tangent lines through a given point $H$ (and in particular, among the prescribed family $\{\ell_k\}$, there cannot be two distinct lines with the \emph{same} forced tangency direction). Since the Lemma pins down a specific tangent line through $H$, namely $\ell_0$, no other $\ell_k$ from this family can coincide with that tangent line.

Hence $\ell_k$ is tangent to $\mathcal{C}$ only for $k=0$.

\section*{Solution 2 (Alternative)}

Fix the point $H$ and the circle $\mathcal{C}$. Consider an arbitrary line $\ell$ through $H$. The intersection points of $\ell$ with $\mathcal{C}$ are the solutions of a quadratic equation (e.g.\ in an affine coordinate system), so $\ell$ is tangent to $\mathcal{C}$ if and only if that quadratic has a double root, i.e.\ its discriminant is $0$. Therefore, among all directions through $H$, at most two directions yield tangents to $\mathcal{C}$.

From the Lemma (the geometric crux of the construction), the specific direction parallel to $AP$ yields a tangent, and the corresponding line is exactly $\ell_0$.

Since the $80{,}000$ lines $\ell_k$ have pairwise distinct directions (their directions differ by multiples of $\pi/80{,}000$), at most one of them can equal the particular tangent line $\ell_0$. Consequently, within the set $\{\ell_k\}$, tangency occurs only for $k=0$.

\section*{Answer}

\[
\boxed{1}
\]

\end{document}
