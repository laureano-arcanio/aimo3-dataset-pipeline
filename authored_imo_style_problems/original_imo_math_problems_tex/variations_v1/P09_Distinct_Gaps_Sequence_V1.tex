\documentclass[11pt]{article}

\usepackage[T1]{fontenc}
\usepackage[utf8]{inputenc}
\usepackage{amsmath,amssymb}
\usepackage[top=2.3cm]{geometry}

\begin{document}
\sloppy

\title{\Huge\bfseries P09 Distinct Gaps Sequence V1}
\author{\Large Laureano Arcanio \quad \large (February 2026)}
\date{}
\maketitle
\section*{Problem}

Fix $C = 3007$.  
Consider a strictly increasing sequence of positive integers
\[
a_0 < a_1 < \cdots < a_m .
\]
Assume that for every $0 \le n \le m-2$,
\[
a_{n+2} - a_n \le 3007.
\]
Define the gaps $d_n = a_{n+1} - a_n$ for $0 \le n \le m-1$, and suppose that the gaps
\[
d_0, d_1, \dots, d_{m-1}
\]
are pairwise distinct.

Find the maximum possible value of $m+1$.

\section*{Solution 1 (Gap analysis)}

Define $d_n = a_{n+1} - a_n$ for $0 \le n \le m-1$.  
Since the sequence $(a_n)$ is strictly increasing, each $d_n$ is a positive integer.

From the given condition,
\[
a_{n+2} - a_n = (a_{n+2} - a_{n+1}) + (a_{n+1} - a_n)
= d_{n+1} + d_n \le 3007
\]
for all $0 \le n \le m-2$.

Because $d_{n+1} \ge 1$, it follows that
\[
d_n \le 3006 \quad \text{for all } n.
\]
Hence all gaps belong to the set $\{1,2,\dots,3006\}$.

Since the gaps are pairwise distinct, we must have
\[
m \le 3006,
\]
and therefore
\[
m+1 \le 3007.
\]

We now show that this bound is attainable.  
Arrange the numbers $1,2,\dots,3006$ in the order
\[
3006,\,1,\,3005,\,2,\,3004,\,3,\,\dots
\]
alternating between the largest unused and smallest unused value.  
For each adjacent pair in this sequence, the sum is at most $3007$.

Let $d_0,\dots,d_{3005}$ be this ordering, and define $a_0 = 1$ and
\[
a_{n+1} = a_n + d_n.
\]
Then $(a_n)$ is strictly increasing, satisfies $a_{n+2} - a_n \le 3007$, and has $3007$ terms.

Thus the maximum possible value of $m+1$ is $3007$.

\section*{Solution 2 (Graph-theoretic interpretation)}

Consider the graph whose vertices are the integers $1,2,\dots,3006$, with an edge between
distinct vertices $x$ and $y$ if and only if
\[
x + y \le 3007.
\]

A sequence of pairwise distinct gaps $d_0,\dots,d_{m-1}$ satisfying
$d_n + d_{n+1} \le 3007$ corresponds exactly to a simple path of length $m-1$ in this graph.
Hence $m$ cannot exceed the number of vertices, which is $3006$, so $m+1 \le 3007$.

The alternating ordering
\[
3006,\,1,\,3005,\,2,\,\dots
\]
defines a Hamiltonian path in this graph, since every adjacent pair sums to at most $3007$.
Therefore a path using all $3006$ vertices exists, giving $m=3006$ and $m+1=3007$.

\section*{Answer}

\[
\boxed{3007}
\]

\end{document}
