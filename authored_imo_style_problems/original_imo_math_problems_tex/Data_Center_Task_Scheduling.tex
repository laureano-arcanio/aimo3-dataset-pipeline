\documentclass[11pt]{article}

\usepackage[T1]{fontenc}
\usepackage[utf8]{inputenc}
\usepackage{amsmath,amssymb}
\usepackage[top=2.3cm]{geometry}

\begin{document}
\sloppy

\title{\Huge\bfseries Line Route with All Distances Different}
\author{\Large Laureano Arcanio \quad \large (February 2026)}
\date{}
\maketitle
\section*{Problem}

A data center operates over $n$ consecutive days and contains $n$ server racks, both indexed
$1,2,\dots,n$.
For each pair $(i,j)$ with $1\le i,j\le n$, there is a unit maintenance task scheduled for
day $i$ on rack $j$.

Due to constraints, exactly one task per day and exactly one task per rack must be left
\emph{unscheduled}.
All remaining tasks must be scheduled using disjoint \emph{rectangular campaigns}:
a campaign is specified by intervals $[a,b]$ of days and $[c,d]$ of racks and schedules
all tasks $(i,j)$ with $a\le i\le b$ and $c\le j\le d$.
No task may be scheduled by more than one campaign.

Let $n=50625$.
Determine the minimum number of campaigns required.

\section*{Solution 1 (Direct combinatorial argument)}

Let the unscheduled task on day $i$ be on rack $\pi(i)$.
Since each rack also has exactly one unscheduled task, $\pi$ is a permutation of
$\{1,2,\dots,n\}$.
Thus the unscheduled tasks form the permutation matrix
$\{(i,\pi(i))\}$, and the campaigns must partition its complement into disjoint
axis-aligned rectangles.

Let $L$ and $D$ be the lengths of the longest increasing and decreasing subsequences of
$\pi$, respectively.
By the Erd\H{o}s--Szekeres theorem,
\[
L\cdot D \ge n.
\]
Since $n=225^2$, we have $\max(L,D)\ge 225$.

A standard extremal tiling argument shows that any increasing subsequence of length $t$
forces at least $n+t-2$ rectangles, and similarly for any decreasing subsequence.
Applying this in both directions yields the lower bound
\[
\text{number of campaigns} \ge n + (L-1) + (D-1) \ge n + 2\cdot 225 - 3.
\]

Now construct a matching configuration.
Partition the $n\times n$ grid into $225\times225$ blocks of size $225\times225$.
Choose the permutation $\pi$ so that exactly one unscheduled task lies in each block row
and block column.
Tile the interior blocks with $225^2$ rectangular campaigns and use $2\cdot225-3$
additional campaigns to isolate the unscheduled tasks along the boundaries.

This uses exactly
\[
n + 2\cdot225 - 3
\]
campaigns, matching the lower bound.

\section*{Solution 2 (Poset and boundary-turn viewpoint)}

View the unscheduled tasks as points $(i,\pi(i))$ in the grid.
Introduce the partial order
\[
(i,\pi(i)) \prec (j,\pi(j)) \iff i<j \text{ and } \pi(i)<\pi(j).
\]
Chains in this poset correspond to increasing subsequences of $\pi$.
By Erd\H{o}s--Szekeres, there exists a chain or antichain of size at least $225$.

Each rectangular campaign has a boundary with only four corners.
As the union of all scheduled tasks must avoid the permutation points, its boundary
must ``turn'' around each point in a long chain or antichain.
These turns cannot be absorbed into fewer rectangles without either covering an
unscheduled task or overlapping campaigns.

Counting the necessary boundary turns in both increasing and decreasing directions
forces at least
\[
n + 2\cdot225 - 3
\]
rectangles in total.
The block construction for $n=225^2$ achieves this bound, so it is optimal.

\section*{Answer}

\[
\boxed{51072}
\]

\end{document}
