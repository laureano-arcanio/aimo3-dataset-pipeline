\documentclass[11pt]{article}

\usepackage[T1]{fontenc}
\usepackage[utf8]{inputenc}
\usepackage{amsmath,amssymb}
\usepackage[top=2.3cm]{geometry}

\begin{document}
\sloppy

\title{\Huge\bfseries Terminal Grid and Direct Routes}
\author{\Large Laureano Arcanio \quad \large (February 2026)}
\date{}
\maketitle
\section*{Problem}


Let $n \ge 3$. In a terminal facility, there are docking nodes indexed by integer pairs $(a,b)$ satisfying
\[
a>0,\qquad b>0,\qquad a+b\le n+1.
\]
A \emph{rail} is any straight line in the plane. A rail is called \emph{standard} if it is parallel to one of the three directions
\[
x=\mathrm{const}, \qquad y=\mathrm{const}, \qquad x+y=\mathrm{const}.
\]
A rail that is not standard is called \emph{express}.

\section*{Solution 1 (Reduction Using the Long-Rail Lemma)}

Assume the following lemma, whose proof can be given using the Alon--F{\"u}redi polynomial method:

\medskip
\noindent
\emph{Long-rail lemma.}  
For every $n\ge4$, any collection of $n$ distinct rails covering all points of
\[
T_n=\{(a,b)\in\mathbb Z^2:\ a>0,\ b>0,\ a+b\le n+1\}
\]
must contain at least one of the three boundary rails
\[
x=1,\qquad y=1,\qquad x+y=n+1.
\]

\medskip
Each of these boundary rails is standard. Removing one such rail and translating the remaining configuration reduces the problem from $n$ to $n-1$ while preserving the number of express rails. Hence the set $\mathcal K_n$ of attainable values of $k$ is the same for all $n\ge3$.

It therefore suffices to determine $\mathcal K$ in the base case $n=3$.

For $n=3$, the set $T_3$ consists of the six points
\[
(1,1),(1,2),(1,3),(2,1),(2,2),(3,1).
\]
One checks directly that:
\begin{itemize}
\item $k=0$ is attainable using the three standard rails $x=1$, $y=1$, and $x+y=4$;
\item $k=1$ is attainable using two standard rails and one express rail;
\item $k=2$ is impossible, since any line through two of $(2,1),(2,2),(3,1)$ is necessarily standard;
\item $k=3$ is attainable using three suitably chosen express rails.
\end{itemize}

Thus
\[
\mathcal K=\{0,1,3\}.
\]
By definition,
\[
N
=10000\cdot |\mathcal K|+\sum_{k\in\mathcal K}k
=10000\cdot 3+(0+1+3)
=30004.
\]

\section*{Solution 2 (Direct Evaluation of the Final Expression)}

From the complete analysis of the problem (using the reduction argument and the corrected base case),
the set of attainable values of $k$ is
\[
\mathcal K=\{0,1,3\}.
\]
Substituting this into the definition of $N$, we obtain
\[
N
=10000\cdot |\mathcal K|+\sum_{k\in\mathcal K}k
=10000\cdot 3+(0+1+3)
=30004.
\]

\section*{Answer}

\[
\boxed{30004}
\]

\end{document}
