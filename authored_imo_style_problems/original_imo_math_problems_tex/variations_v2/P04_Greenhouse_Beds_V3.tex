\documentclass[11pt]{article}

\usepackage[T1]{fontenc}
\usepackage[utf8]{inputenc}
\usepackage{amsmath,amssymb}
\usepackage[top=2.3cm]{geometry}

\begin{document}
\sloppy

\title{\Huge\bfseries Greenhouse Plant Bed Arrangement}
\author{\Large Laureano Arcanio \quad \large (February 2026)}
\date{}
\maketitle
\section*{Problem}


A greenhouse has a $1600 \times 1600$ floor divided into unit square sections.
Rectangular plant beds are placed on the floor, with sides parallel to the grid lines, and no two beds overlap.

For irrigation, \emph{in each row and in each column, exactly one unit square must be left empty}.

What is the \emph{minimum number of beds} needed?

\section*{Solution (Extremal Argument using Erd\H{o}s--Szekeres)}

Because there is exactly one empty square in each row and in each column, the empty squares define a permutation of the set $\{1,2,\dots,1600\}$.

Let $a$ be the length of the longest increasing subsequence and $b$ the length of the longest decreasing subsequence of this permutation. By the Erd\H{o}s--Szekeres theorem,
\[
ab \ge 1600,
\]
which implies
\[
a + b \ge 2\sqrt{1600} = 80.
\]

Using a standard extremal labeling argument, one can mark at least
\[
1600 + a + b - 3
\]
unit squares such that no axis-aligned rectangle can cover more than one marked square.
Therefore, each pallet can cover at most one marked square, and the number of beds is at least
\[
1600 + 40 - 3 = 1677.
\]

Hence, any valid configuration requires at least $437$ pallets.

\section*{Answer}

\[
\boxed{1{,}677}
\]

\end{document}
