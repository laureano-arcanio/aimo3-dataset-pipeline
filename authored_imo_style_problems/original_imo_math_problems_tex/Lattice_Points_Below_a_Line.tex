\documentclass[11pt]{article}

\usepackage[T1]{fontenc}
\usepackage[utf8]{inputenc}
\usepackage{amsmath,amssymb}
\usepackage[top=2.3cm]{geometry}

\begin{document}
\sloppy

\title{\Huge\bfseries Lattice Points Below a Line}
\author{\Large Laureano Arcanio \quad \large (February 2026)}
\date{}
\maketitle
\section*{Problem}

Consider the integer lattice in the plane.  
Let \(\alpha\) be a real number, and consider the line \(y=\alpha x\).

For each positive integer \(k\), let \(h_k\) denote the number of integer points \((k,y)\) with \(y\ge 0\) that lie strictly below the line \(y=\alpha x\).  
Equivalently,
\[
h_k=\lfloor k\alpha\rfloor.
\]

For each \(n\ge1\), define
\[
H_n=\sum_{k=1}^n h_k.
\]

Call \(\alpha\) \emph{admissible} if \(H_n\) is divisible by \(n\) for every positive integer \(n\).

Let \(B=5998\).  
Determine the number of admissible real numbers \(\alpha\) in the interval \([0,B]\).

\section*{Solution 1 (Direct approach)}

For \(\alpha=2m\), where \(m\) is an integer, we have
\[
h_k=\lfloor 2mk\rfloor=2mk,
\]
and therefore
\[
H_n=\sum_{k=1}^n 2mk = 2m \cdot \frac{n(n+1)}{2} = m n(n+1),
\]
which is divisible by \(n\) for all \(n\). Hence every even integer \(\alpha\) is admissible.

Conversely, suppose \(\alpha\) is admissible. By considering the divisibility of
\[
H_n=\sum_{k=1}^n \lfloor k\alpha\rfloor
\]
for all \(n\), a standard pairing argument comparing the terms
\(\lfloor k\alpha\rfloor\) and \(\lfloor (n+1-k)\alpha\rfloor\) shows that the associated carry terms must be rigidly constrained. This forces \(\alpha\) to be an integer. Substituting \(\alpha=m\in\mathbb{Z}\) gives
\[
H_n = m\frac{n(n+1)}{2},
\]
which is divisible by \(n\) for all \(n\) if and only if \(m\) is even. Thus the admissible values of \(\alpha\) are exactly the even integers.

The admissible values in \([0,5998]\) are
\[
0,2,4,\dots,5998,
\]
whose count is
\[
\frac{5998}{2}+1=3000.
\]

\section*{Solution 2 (Average-value viewpoint)}

Define
\[
A_n=\frac{H_n}{n}.
\]
By assumption, \(A_n\) is an integer for every \(n\). Writing
\[
H_n=\sum_{k=1}^n (k\alpha-\{k\alpha\})=\alpha\frac{n(n+1)}{2}-\sum_{k=1}^n \{k\alpha\},
\]
we obtain
\[
A_n=\alpha\frac{n+1}{2}-\frac{1}{n}\sum_{k=1}^n \{k\alpha\}.
\]
Since \(0\le \{k\alpha\}<1\), the second term lies in \([0,1)\), so \(A_n\) lies in an interval of length less than \(1\) whose right endpoint is \(\alpha\frac{n+1}{2}\).

For \(A_n\) to be an integer for all \(n\), this forces \(\alpha\frac{n+1}{2}\) to be consistently arbitrarily close to integers, which implies that \(\alpha\) itself must be an integer. Substituting \(\alpha=m\in\mathbb{Z}\) reduces the condition to
\[
A_n = m\frac{n+1}{2}\in\mathbb{Z}\quad\text{for all }n,
\]
which holds if and only if \(m\) is even. Thus \(\alpha\) must be an even integer.

Counting even integers in \([0,5998]\) again yields \(3000\).

\section*{Answer}

\[
\boxed{3000}
\]

\end{document}
